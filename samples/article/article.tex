\documentclass[10pt,twocolumn]{article}

% -------------------------------------------------------------
% Package Loading
% -------------------------------------------------------------
% NAG: check for outdated packages
\usepackage[l2tabu, orthodox]{nag}

% GEOMETRY: set page size
\usepackage[head=0.125in,top=1.5in,bottom=1in,inner=1in,outer=1in]{geometry}

% MATHPTMX Times for roman text and math (family ptm)
\usepackage{mathptmx}

% Helvetica for sans serif (family phv)
\usepackage{helvet}

% Courier for typewriter font (family pcr)
\usepackage{courier}

% ams math
\usepackage{amsfonts,amssymb,amsmath}

% gensymb. Required for degrees symbol
\usepackage{textcomp}  % load before gensymb for \micro and \perthousand
\usepackage{gensymb}

% better tables
\usepackage{booktabs}

% graphics
% can set the option 'final' to override 'draft' status
\usepackage{graphicx}

% For formatting the bibliography
\usepackage[sort]{natbib}

% use fonts of type T1
\usepackage[T1]{fontenc}

\usepackage{cmap}

% colours
\usepackage{color}
\usepackage{xcolor}
\definecolor{prettyblue}{RGB}{0, 121, 193}

% set languages
\usepackage[english]{babel}
    \addto{\captionsenglish}{\renewcommand{\bibname}{References}}
    \addto\captionsenglish{\renewcommand{\contentsname}{Table of Contents}}

% To stop hyphenation in titles etc
\usepackage{hyphenat}

% spacing
\usepackage{setspace}
\usepackage{parskip}

\usepackage{etoolbox}

\usepackage[newcommands,document]{ragged2e}

% improved table of contents and list of figures/tables
\usepackage{tocloft}

% remove numbering on bibliography but still allow in TOC
\usepackage[nottoc, notlot, notlof]{tocbibind}

\usepackage[format=plain,
	labelformat=simple,
	font={small,sf,bf},
	labelfont={small,sf,bf},
	textfont={small,sf,bf},
	indention=0cm,
	labelsep=period,
	justification=centering,
	singlelinecheck=true,
	tableposition=top,
	figureposition=bottom]{caption}
\usepackage{subcaption}

\usepackage[%
	linktocpage,
	colorlinks,
	linktoc=all,
	linkcolor=blue,
	citecolor=blue,
	menucolor=blue,
	urlcolor=blue,
	pdfborder={0 0 0}]{hyperref}

% -------------------------------------------------------------
% Set up listings
% -------------------------------------------------------------
% code listings
%\usepackage{listings}

%\lstnewenvironment{code}[1][firstnumber=\themain,name=main]
%	{\lstset{language=[LaTeX]TeX,	% the language of the code
%		basicstyle=\small\sffamily,
%		numbers=left,                   % where to put the line-numbers
%		numberstyle=\tiny\color{gray},  % the style that is used for the line-numbers
%		stepnumber=1,                   % the step between two line-numbers. If it's 1, each line will be numbered
%		numbersep=5pt,                  % how far the line-numbers are from the code
%		backgroundcolor=\color{white},  % choose the background color. You must add \usepackage{color}
%		showspaces=false,               % show spaces adding particular underscores
%		showstringspaces=false,         % underline spaces within strings
%		breaklines=true,                % sets automatic line breaking
%		breakatwhitespace=true,        % sets if automatic breaks should only happen at whitespace
%		keywordstyle=\color{blue},      % keyword style
%		commentstyle=\color{gray},   % comment style
%		stringstyle=\color{green},      % string literal style
%		#1
%		}		            % if you want to add more keywords to the set
%	}

% -------------------------------------------------------------
% Tooltips
% -------------------------------------------------------------

% Enable tooltips
%\usepackage[linewidth = 1]{pdfcomment}

% -------------------------------------------------------------
% Tagging Loading
% -------------------------------------------------------------
\usepackage[tagged,highstructure]{accessibility}

% new commands
\newcommand{\fn}[1]{\emph{#1}}
\newcommand{\packagename}[1]{\textbf{\emph{#1}}}
\newcommand{\envname}[1]{\textbf{\texttt{#1}}}

\author{Andy Clifton}
\title{An example of the `accessibility' style file in use}

\begin{document}

\maketitle

%\pagenumbering{roman}
\section*{Abstract}
Structured and tagged PDFs are required to meet modern corporate and governmental standards for document accessibility. PDFs that are created with core \LaTeX\ are not tagged or structured, making it difficult to use \LaTeX\ in a corporate or government environment. This document explains how \LaTeX\ can be used to prepare documents that pass such tests.

This document is intended to be used as a test case as it contains most of the elements of a technical \LaTeX\ document, including horrific formatting, custom fonts, complex document structures, lists, equations, figures and code listings.

%\pagenumbering{arabic}
\tableofcontents
\listoffigures
\listoftables

\section{Introduction}
The de-facto standard for scientific publishing is \LaTeX. \LaTeX\ is often preferred over WYSIWYG word processors for technical documents because of the relatively simple file format that can be shared across users on many different platforms, and the ease of formatting a document for journal publication.

However, one issue with using \LaTeX\ is document \emph{accessibility}. Accessibility is important for documents produced by federally-funded organizations: since the US Congress passed the 1998 Section 508 Amendment to the Rehabilitation Act of 1973, it has been a requirement that all federally-funded documents are accessible to people with disabilities.

An accessible PDF has several characteristics:

\begin{itemize}
\item All of the document content has been tagged
\item It is possible to define a reading order based on those tags
\item Images and links are given alternate text descriptions
\item Tables are tagged, so that the table structure can be established
\item Unicode descriptions of all characters are required
\end{itemize}

A document that has these characteristics is often referred to as being `508 compliant'. As 508-compliance is often judged using automated tests on the \fn{.pdf} file, there is no option to work around this by using careful text descriptions of figures, for example.

In this document, I explain how \LaTeX can be generated using the \packagename{accessibility} style file.

My goal is that this will be a `living' document and template that can be updated as we gain new insight into this process.

\section{Some more text}

Table \ref{Tab:Packages} lists the packages that are included in this demonstration article. These packages often call other packages, so this is not an exhaustive list.

\begin{table*}
\centering
\caption[Packages explicitly loaded for this document]{Packages explicitly loaded for this document}
\label{Tab:Packages}
\begin{tabular*}{\textwidth}{lll}
\toprule
Packages & options & functionality\\
\midrule
nag & & checks that packages are up to date and looks for bad habits in \LaTeX\ code.\\
geometry & & sets page size and margins \\
mathptmx& & changes fonts\\
helvet& & changes fonts\\
courier& & changes fonts\\
amsfonts, amssymb & & supplies fonts that are useful for mathematics\\
booktabs & & \\
graphicx & &graphics handling, including \emph{.eps} figures (see Section \ref{sec:Alttext})\\
natbib & sort &handles citations and allows the \verb+\cite+, \verb+\citep+ and \verb+\citet+ citation commands.\\
fontenc & T1 &\\
xcolor & & \\
babel & english & \\
subcaption & & provides the \texttt{subfigure} environment to produce sub figures \\
hyphenat & &\\
setspace & &\\
parskip & & \\
toclof & subfigure &\\
toclifbind & nottoc, notlot, notlof &\\
todonotes & & inline and margin to-do notes \\
listings & &\\
caption & &\\
cmap & & \\
pdfcomment & & tool-tips. Also calls the package \packagename{hyperref} \\
\bottomrule
\end{tabular*}
\end{table*}

\subsection{Accessibility support}
\LaTeX\ does not prepare a structured PDF document directly. Instead, we use the \packagename{accessibility} package to do this for us. This generates a tagged PDF that passes most automated document tests.

\subsubsection{Alternative text}\label{sec:Alttext}
Alternative text, or `Alt text', is a textual description of an equation, link or figure that can be used to replace the visual information in that element. This is often seen as a text `pop-up' in PDF readers. Alt text can be added after the PDF is compiled using a PDF editor such as Adobe's Acrobat Pro. Alternatively -- and probably best for ensuring that the final document is what the author intended -- it can be generated from within the source document using the \envname{pdftooltip} environment from the \packagename{pdfcomment} package. 

For example, Figure \ref{fig:AltTextImages} has been labeled with a tool tip.

\begin{figure*}
          \begin{subfigure}[b]{.55\linewidth}
            \centering
		%{\pdftooltip{\includegraphics[height=2.5in]{Chick1}}{A bright yellow toy model of a chick}}
		\includegraphics[height=2.5in]{Chick1}
            \caption{A chick.}\label{fig:ChickWithAltText}
          \end{subfigure}%
          \begin{subfigure}[b]{.55\linewidth}
            \centering
		%{\pdftooltip{\includegraphics[height=2.5in]{Chick1}}{A second image of a bright yellow toy model of a chick}}
		\includegraphics[height=2.5in]{Chick1}
            \caption{Another chick}\label{fig:ChickWithAltText2}
          \end{subfigure}
          \caption{Test images}
          \label{fig:AltTextImages}
\end{figure*}

Note that the \texttt{subfig} and \texttt{subfigure} packages are deprecated and so subfigures are implemented using the \texttt{subcaption} package. The \texttt{subcaption} package appears to be the most frequently maintained package at this time, and contains the same functionality as the \texttt{subfig} and \texttt{subfigure} packages.

As a further demonstration that tooltips actually work, passing the pointer over the following equation should reveal a pop-up:

\begin{equation}
	%{\pdftooltip{a^2+b^2=c^2}{An equation}}
	a^2+b^2=c^2
\end{equation}

The \packagename{accessibility} package includes an \verb+\alt{}+ environment which is intended to create a tool tip. Although it has been included in the source of the next equation, it does not currently work.

\begin{equation}
	%\alt{a squared plus b squared equals c squared}
	a^2+b^2=c^2
\end{equation}

\subsubsection{Problems with embedded fonts}
One requirement of passing automated tests for accessibility is that fonts must be embedded in the the final PDF. You can check the PDF for embedded fonts using a PDF viewer. For example, in Adobe Acrobat Reader, look at the `fonts' tag of the document properties. If any fonts are not shown as being an \emph{embedded subset}, you need to try again. 

Encapsulated postscript figures are particularly prone to having undefined fonts. Check by compiling your document in draft mode, and seeing if the fonts are still present in the output PDF. To fix this problem, you could consider changing the \emph{.eps} file to a \emph{.png}. If you wish to do this `on the fly', you could use this approach in your preamble:

%\begin{code}[backgroundcolor=\color{lightgray!20}]
%	%\usepackage{epstopdf}
%	\epstopdfDeclareGraphicsRule{.eps}{png}{.png}{convert eps:\SourceFile.\SourceExt png:\OutputFile}
%	\AppendGraphicsExtensions{.png}
%\end{code}

\subsection{Including code listings}
The \packagename{listings} package is one of several packages that can be used to typeset source code, and is used in this document. It seems to work.

\section{A template}
The code used to produce this document is available from \href{https://github.com/AndyClifton/accessibility}{https://github.com/AndyClifton/accessibility}. 

\section{Problems with this approach}
Well, there are lots. If you find any, please use GitHub's issue tracking to report these. You can find the current list of issues at \url{https://github.com/AndyClifton/accessibility/issues}.

\section{Conclusions}
A \LaTeX\ style file was created in order to generate accessible \fn{.pdf} files with \LaTeX\. Accessible \fn{.pdf} files are compiled directly from the \LaTeX source code.

\section*{Acknowledgements}
This document benefitted from contributions to the website, \url{http://tex.stackexchange.com/}.

Babett Schalitz produced the original \packagename{accessibility} package in 2007. That package was oriented towards KOMA-script documents. It was not accepted by CTAN and was subsequently not available to the \LaTeX community.

Babett Schalitz provided me with a copy of the original \packagename{accessibility} package in May 2019 and asked me to take up maintenance with a goal of submitting it to CTAN. This document is intended to support that effort. I am extremely grateful for all of Babett's work!

\end{document}
% bibliography
\cleardoublepage
\bibliographystyle{plainnat}
%\bibintoc
\label{sec:Bib}
\bibliography{bibliography}

\end{document}