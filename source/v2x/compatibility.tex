%FILE: compatibility.tex


%======================================================================
%	Erkennen des Bearbeitungsmodus f�r weitere Paketoptionen
%======================================================================
\RequirePackage{ifthen}

\newif\ifhtm
\ifx \HCode\undefined \htmfalse \else	\htmtrue \fi

\newif\ifpdf
\ifx \pdfoutput\undefined
	\pdffalse 		%latex in DVI mode
\else
	\ifthenelse{\number\pdfoutput<1}%
 	{\pdffalse} 	%pdflatex in DVI mode
 	{\pdftrue} 		%pdflatex in PDF mode
\fi

%======================================================================
%		PDF <=> DVI <=> XHTML
%======================================================================
\ifhtm%
		% Bearbeitung mit tex4ht
		\usepackage{graphicx}%
		\usepackage[tex4ht]{hyperref}  %
		\newcommand{\thead}[1]{#1}%
		\newcommand{\lineheight}{12pt}%
\else%
	\ifpdf%
		%Bearbeitung mit pdftex im PDF-Modus
		\usepackage[pdftex]{graphicx}%
		\usepackage[pdftex]{hyperref}%
		\pdfcompresslevel=0%			Damit wird die PDF-Quelldatei lesbar
		\pdfoptionpdfminorversion=6%	Bestimmt die PDF - Version der Ausgabe
	%\pdfadjustspacing=0%				0, 1 oder 2 �nderung nicht erkannt
		\newcommand{\lineheight}{12pt}%
		\AtBeginDocument{\usepackage[untagged, highstructure]{accessibility}}	% Makros in accessibility.sty
	\else%
		%Bearbeitung mit latex oder pdftex im DVI-Modus
		\usepackage{graphicx}%
		\usepackage{hyperref}%
		\newcommand{\alt}[1]{}%
		\newcommand{\thead}[1]{{\bfseries#1}}%
		\newcommand{\lineheight}{12pt}%
	\fi%
\fi%

% Einstellungen des Hyperref-Paketes
	\hypersetup{colorlinks=true,%
	linkcolor=sun1,%
	citecolor=sun1,%
	filecolor=sun1,%
	menucolor=sun1,%
	%pagecolor=sun1,%
	urlcolor=sun1%
	}%
